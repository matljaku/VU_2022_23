
%% === nezbytné balíčky:
\usepackage[T1]{fontenc} % kódování písma
%\usepackage{dsfont} %citace
%\usepackage[nottoc]{tocbibind}      % citace
\usepackage[utf8]{inputenc}     % vstupní znaková sada tohoto dokumentu: UTF-8
\usepackage[nottoc]{tocbibind}
\usepackage{makecell}
\usepackage{nameref}
\usepackage{lmodern}
\usepackage{xargs}
\usepackage[title]{appendix}
\usepackage{import}
\usepackage{pgfplots}
\pgfplotsset{compat=1.18, width=10cm}
\usepackage{subcaption}
\usepackage[font=scriptsize]{caption}
%\usepackage[cp1250]{inputenc}  % vstupní znaková sada tohoto dokumentu: Windows 1250
%\usepackage[latin2]{inputenc}  % vstupní znaková sada tohoto dokumentu: ISO Latin 2

\usepackage[czech]{babel} % česky psaná práce, typografická pravidla. Překládejte pomocí "latex.exe" nebo "pdflatex.exe"
%\usepackage{czech} % česky psaná práce. Překládejte pomocí "pdfCSlatex.exe" ("cslatex.exe" asi bude mít problém s balíkem geometry)

\usepackage[a4paper, hmarginratio=3:2]{geometry} % využití A4 stránky a nastavení okrajů (u vazby bude širší)

\usepackage{pdfpages} % pokud nemáte formulář "Zadání bak./dipl. práce" naskenovaný jako PDF, tak ZAKOMENTUJTE
\usepackage[hidelinks]{hyperref} % v PDF budou klikací odkazy ("hidelinks" je nebude rámovat)

%% === balíčky, které se mohou hodit:
%\usepackage{encxvlna} % postará se o spojky a předložky, které dle českých pravidel nesmí být na konci řádku. Dokumentace: http://texdoc.net/texmf-dist/doc/generic/encxvlna/encxvlna.pdf (chová se správně k "vnitřku" listings?)

\usepackage{graphicx} % balíček pro vkládání rastrových grafických souborů (PNG apod.)
%\usepackage{epsfig} % balíčky pro vkládání grafických souborů typu EPS
\usepackage{float} % rozšířené možnosti umístění obrázků

\usepackage{caption} % pro popisky obrázků, tabulek atd.

\usepackage{tabularx} % rozšířené možnosti tabulek
%\usepackage{tabu} % jiný balík pro rozšířené možnosti tabulek

\usepackage{listings}  % balíček vhodný pro ukázky zdrojového kódu v~textu práce/příloh. Nutno nastavit! http://ftp.cvut.cz/tex-archive/macros/latex/contrib/listings/listings.pdf
\usepackage{amsmath} % balíček pro pokročilou matematickou sazbu
%\usepackage{color} % pro možnost barevného textu
%\usepackage{fancybox} % umožňuje pokročilé rámečkování
\usepackage{fancyhdr} % Tvorba záhlaví a zápatí
\usepackage{adjustbox} %umožnuje upravit velikost tabulek atd.
\usepackage[scientific-notation=true]{siunitx}
\sisetup{exponent-product = \cdot, output-decimal-marker={,}}
%\usepackage{index} % nutno použít v případě tvorby rejstříku balíčkem makeindex
%\usepackage{xcolor} % balíček pro barvy
%\newindex{default}{idx}{ind}{Rejstřík} % zavádí rejstřík v případě použití balíku index
